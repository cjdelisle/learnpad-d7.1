\chapter{Development process}
\label{ch:development-process}

\section{Documentation}
\label{sec:documentation2}

There are a few occasions when documentation will be needed such as architectural
design and configuration.

\subsection{Architecture design}
\label{sec:architecture-design}

According to the Agile Development Methodology, architectural design should not be set in
stone and changes can be made as experience is gained. Therefor architectural design
decisions need not be perfect the first time, better to have something quickly which
satisfies the obvious use cases than to try to project every possible need.

Discussion of architectural design should take place on the mailing list or in logged
chatroom where there is a record of the thoughts and reasoning. As ideas solidify and
consensus forms, they should move to the wiki where they become official documentation.

For producing diagrams, a server is available for collaborative work using MagicDraw.
When applicable, MagicDraw can be used for development of ideas on concert with the
mailing list. To minimize the number of places where documentation resides, ideas and
diagrams which have solidified should be exported and placed in the wiki or in the source
repository.

\subsection{Interfaces}
\label{sec:interfaces}

Two main types of interfaces are Application Programming Interface (API): the way in
which components communicate, and User Interface (UI): the way in which a component
communicates with a human. In either case it is important to clearly document interfaces
and put a link to this documentation in the \texttt{readme.md} file
(see Section~\ref{sec:source-structure}).
For UI development, use of mockups is highly recommended.

\subsection{Configuration and Deployment}
\label{sec:deployment-process}

Since the system will be automatically installed on a live test server
(see Section~\ref{sec:live-snapshot}), each component requiring configuration must contain
default configuration.

Documentation of the configuration should be done in the configuration files. If more
documentation is necessary, there must be URLs in the configuration file which tell
administrators where to find the additional information.

\section{Roadmap and Development Milestones}
\label{sec:roadmap}

There is two different types of roadmaps.

\subsection{Global roadmap}
\label{sec:global-roadmap}

The global roadmap will be a broad picture of the project, based on the deliverable
deadlines. Releases will take place every three months or six Development Milestones.
In the final two week period before the release date, the most recent Development
Milestone will be manually tested by each partner in order to verify that the
features for which this partner is responsible work well. At the beginning of this
two week development cycle, a stable branch will be created and only critical and
"safe" bug fixes will be accepted to the branch prior to release.

\subsection{Partner roadmaps}
\label{sec:partner-roadmaps}

Each component will have a responsible partner and this partner must provide a roadmap
detailing timeframes and objectives for the components under their responsibility. At any
given time, a roadmap should contain goals for the next three months.
Development Milestones will take place every two weeks and each two weeks there will be
a roadmap meeting to check in on progress and set new goals. Each partner's roadmap
should be contained in a document on the wiki where it can be reviewed at each bi-weekly
roadmap meeting.

\section{Source management}
\label{sec:source-management}

All source code will be stored in a single repository using a distributed version control
system such as git. To simplify collaboration, each component will be developed in one
sub-folder of a \texttt{components} directory. The source will be published on a public
server such as GitHub, where it can be cloned or viewed with a web browser.

\subsection{Source structure}
\label{sec:source-structure}
The partner in charge of managing development of a component will be free to organize the content
of this sub-folder in any way as long as a few standardized files are contained within.

\begin{itemize}
\item \texttt{readme.md} This file will contain the entry-point to all documentation of this
component. At the very least, this file must contain contact information for the manager of
the component and a link to their roadmap (see: Section~\ref{sec:roadmap}).
\item \texttt{build} This will be a script file which is executable on the \emph{target system}
(see Section~\ref{sec:continuous-integration}) and will build the component.
\item \texttt{builddeps.txt} This \emph{optional} file will contain a list of packages which
need to be installed using the \emph{target system's} package manager (EG: apt-get) in order for
the component to be built.
\item \texttt{out} This folder will be created by the \emph{build} script and will contain the
result of the build, everything that is necessary for this component to run.
\item \texttt{out/etc} This folder, named after the UNIX \texttt{/etc} configuration folder will
contain all of the configuration files necessary for the component.
\item \texttt{out/start} This will be a script file which initiates the completed component, it will
expect it's configuration to reside in the \emph{out/etc} folder and must be capable of starting
the component with no manual intervention.
\item \texttt{out/stop} This script will behave similarly to \emph{start} but will stop the running
component.
\item \texttt{out/rundeps.txt} This \emph{optional} file will contain a list of package names,
like \texttt{builddeps.txt}, but with packages necessary for the component to \emph{run}.
\end{itemize}

\subsection{Source development flow}
\label{sec:source-development-flow}

We will use the \emph{git flow}\cite{kreeftmeijer:2010} development model. The main repository will
contain a \texttt{master} and a \texttt{develop} branch. Every developer should work on his
own fork of the project then submit pull requests (see Section~\ref{sec:source-repository})
which will be checked by integrators for merging with the \texttt{develop} branch of the main
repository. Then, for each Development Milestone or release, the \texttt{develop} branch will
be merged onto the \texttt{master} branch. Thus the \texttt{master} branch will always contain
the most recent Development Milestone version.

\section{Building}
\label{sec:building}

As seen in the Figure~\ref{fig:development-workflow}, a continuous integration will support the
development of the platform. That means that every service must be automatically buildable.
As seen in Source Management (see Section~\ref{sec:source-management}), each service will be a
folder in the source repository in order to maximize freedom in the management of development
activities (tools used, subdirectories, etc.). To simplify the overall build, each folder must
contain a standardized script for building the component in that folder how that script manages
the build is an implementation detail.

At the root of the repository, a global builder will parse all of the services to build them.
If specific deployment procedures must be done, they should also appear in this build process.

\section{Testing}
\label{sec:testing}

Unit tests may be run as part of the build process but integration tests which require the
\learnpad platform in a running state must be handled differently. The exact methodology
remains a topic for further research but possible solutions include a special sub-folder
for all integration tests or a set of integration tests included in each component and built
separately from the normal build.

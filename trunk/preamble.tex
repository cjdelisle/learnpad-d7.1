\chapter{Preamble}
\label{ch:preamble}

\section{SOA architecture}
\label{sec:soa-architecture}

The \learnpad project involves 9 partners, each of whom will produce one or more functionalities
in this architecture. To ease the collaboration between partners, we need to delimit the scope
for each functionality in order to precisely define the interactions needed. A solution to manage
this problem is encaspulation of functionalities which delimits the scope with inputs and outputs.
Encapsulation as services leads us to a \emph{Service-Oriented Architecture} also known as SOA.

SOA is a loosely coupled architecture which means that:
\begin{itemize}
	\item each service delimits a scope of functionalities
	\item each interaction between services is defined as a public interface
\end{itemize}

\section{Source management}
\label{sec:source-management}

In \learnpad projects, most developments will be distributed under an Open Source license.
All source code developed in the consortium will be available publicly on a source versioning
system such as GitHub. Integration of proprietary products into the project where the source
is not available will be handled on a case by case basis.

\subsection{External service}
\label{sec:external-service}

In some cases, functionality can be deployed as a public service, accessible from any internet
connection (for example, REST API). In this case, the service can be developed independantly of
the main \learnpad platform and no file of any kind would be pushed on the source repository.
The development of this service would meet the requirements of a pre-defined roadmap and the
service must be available during test procedures.

\subsection{Internal service}
\label{sec:internal-service}

In cases when it is easier to create the service as a component inside of the \learnpad platform
(for example: a library or an executable), only the resulting binary file will be uploaded to the
public source repository. The development of this component can be independant of the main
development effort but it will still need to meet the requirements of a pre-defined roadmap.

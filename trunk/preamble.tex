\chapter{Preamble}
\label{ch:preamble}

\section{SOA architecture}
\label{sec:soa-architecture}

The LearnPAd project is involving 9 partners, each of them producing one or more functionalities in this architecture.  To ease the collaboration between partners, we need to delimit the scope for each functionality in order to precisely define the interactions needed.  A solution to manage this problem is encaspulation of functionalities which delimits the scope with inputs and outputs.  Encapsulation as services leads us to a Service-Oriented Architecture also known as SOA.

SOA is a loosely coupled architecture which means that:
* each service delimits a scope of functionalities
* each interaction between services is defined as a public interface

\section{Source management}
\label{sec:source-management}

In LearnPAd projects, most developments will be licensed as open source.  That means that every source code developed in the consortium will be available publicly, mostly on a source versioning system (for example, GitHub, BitBucket or a svn server).
However, some of them must be licensed as proprietary source code which will have consequences on our development process.  We still must be able to build and deploy the LearnPAd platform but we cannot access sources in this case.  Depending on the situation, we may have different solutions.

\subsection{External service}
\label{sec:external-service}

In some cases, the functionality can be deployed as a public service, accessible from any internet connection (for example, REST API).  In this case, the service can be developed independantly of the main branch of LearnPAd platform; no file of any kind would be pushed on the source repository of the LearnPAd project.  The development of this service will meet the requirements of a pre-defined roadmap and the service should be available during test procedures.

\subsection{Internal service}
\label{sec:internal-service}

In some other cases, it may easier to create the service as a component (for example, a library, an executable).  In this case, only the resulting binary file will be pushed on the source repository of LearnPAd project; only stable versions of this binary files will be pushed.  The development of this service can still be developed independantly of the main branch of LearnPAd platform but will meet the requirements of a pre-defined roadmap.

\chapter{Integration plan}
\label{ch:integration-plan}

\section{Releasing process}
\label{sec:releasing-process}

Each month, a version of the platform will be released.
These short releasing deadlines don't expect stable version each time (see Roadmaps in Section~\ref{sec:roadmap}) but at least version without major bugs.
The goal of these short releasing deadlines are:
* make the platform improve step by step
* giving clear and small objectives
* track easily the work
* detect problems early

\section{Roadmap}
\label{sec:roadmap}

There is 2 different kind roadmaps.
\subsection{Global roadmap}
\label{sec:global-roadmap}

A global roadmap of the platform must be provided.
It will be mainly based on the deliverable deadlines with a cycle of multiple unstable release to finally reach a final stable release each 3 months.
This means that each version of the \learnpad platform will go through at the end of the first month milestone; then a release candidate, one month later; and finally, a stable version after 3 months.
This process will cycle during the whole project.

Roadmaps may be rediscussed during the development, considering problems, locks or important changes in architecture for example.
This means that a roadmap must contains a clear view of the next releasing cycle and only guidelines that may be adapted for the releases after.

\subsection{Services roadmaps}
\label{sec:services-roadmaps}

Each service will also have his own roadmap.
These roadmaps will decribes when each functionality must be included in the main \learnpad platform.
Of course, these roadmap will be mapped onto the main roadmap in order to introduce new functionalities during the cycle of the milestone and possibly for the release candidate.
The last month should be reserved for stabilization of the platform.

As for the global roadmap, services roadmaps may be adapted during the process of development.
